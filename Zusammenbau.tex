Dieser Teil wird beschreiben, wie der Schuldrucker zusammen gebaut wurde, was justiert werden musste, wie die Elektronik angeschlossen ist etc.
Er kann somit teilweise auch zum Troubleshooting verwendet werden, dient allerdings hauptsächlich der Dokumentation.

Der Drucker, mit dem sich dieses Handbuch befasst, ist ein "`GeeTech Prusa i3"'. Dies ist ein "`No-Name"' DIY-Kit zum Zusammenbau eines 3D-Druckers, und kann sich daher jederzeit vom Materialinhalt ändern. Sollte man dementsprechend auf Basis dieses Handbuches seinen eigenen Drucker zusammen bauen wollen, so muss man damit rechnen dass einige Dinge nicht exakt diesen Beschreibungen entsprechen. Es ist immer besser, im Internet nach der aktuellsten Bauanleitung zu suchen, doch selbst diese können eventuell nicht aktuell sein. 3D-Drucker-Zusammenbau ist meist verbunden mit Improvisation und Intuition.

Dies gesagt kann die verwendete, ausführlichere Bauanleitung des Herstellers hier gefunden werden:\\
https://goo.gl/DmXHE7
\begin{center}
\includegraphics{Bilder/Anleitung_QRCode.png}
\end{center}
Im folgenden werden durchaus einige der Bilder des Tutorials für bessere Verständlichkeit genutzt. Diese sind deutlich an den schwarzen Teilen des Druckers erkennbar. Der schulische Drucker hat durchsichtige Acrylglas-Platten, und kann so unterschieden werden.


\subsubsection{Schritt 0: Das Material}
Nach öffnen der Boxen standen uns folgende Materialien zur Verfügung:
\includegraphics[clip=true,trim=230 100 320 100,angle=90,width=\textwidth]{Bilder/Material_1.jpg}
Enthalten hierin waren:
\begin{itemize}[noitemsep]
\item 4x NEMA17 Schrittmotoren
\item Die Kupplungen (Verbindungelemente zwischen Schrittmotor und Z-Achse)
\item GT2 Pulleys
\item X-Achen-Pulley-Verbinder (blaues Plastikteil)
\item Das GeeeTech GT2560 Steuerboard
\item Kabelbinder und Spiralschlauch
\item Lüfter und passende Kabel
\item Verschiedene Schrauben (M3)
\item Linearlager (8mm Innendurchmesser)
\item Teile für die Filamentrollenhalterung (Plastikrohre)
\end{itemize}
\vspace{\linewidth}
\includegraphics[clip=true,trim=410 100 150 100,angle=90,width=\textwidth]{Bilder/Material_2.jpg}
In dieser Packung enthalten waren:
\begin{itemize}[noitemsep]
\item Die verschiedenen Acryl-Platten
\subitem 4x Y-Achsen-Platten
\subitem XZ-Achsen-Frame
\subitem 2x XZ-Y-Verbinder-Platten
\subitem Heizbett-Platte
\subitem 2x Z-Achsen-Motor-Befestigungen
\subitem Die Platten des Filamentrollen-Halters
\subitem Sowie das LC-Display-Befestigungsmaterial

\item Die verschiedenen Linear-Stangen
\subitem Glatte Stahlstangen verschiedener Länge
\subitem 2x M8-Gewindestangen für die Z-Bewegung
\subitem 2x M8-Gewindestangen für den Y-Achsen-Aufbau
\end{itemize}
\vspace{\linewidth}
\includegraphics[clip=true,trim=260 180 230 160,width=\textwidth]{Bilder/Material_3.jpg}
Hier zu finden sind:
\begin{itemize}[noitemsep]
\item Das Netzteil, inklusive Stromkabel und Verbindungskabel mit Schalter
\item Den Extruder-Aufbau (mitsamt Motor, Hotend, Düse etc.)
\item Das Drucker LC-Display, über welches sich dieser steuern lässt, bzw. mit dessen Hilfe über SD-Karte gedruckt werden kann.
\item Das beiliegende Schraubendreher-Set
\item Das USB-Kabel zur Verbindung des GT2560 mit einem geeigneten Computer bzw. Pi
\item Einige Metallteile, z.B. die X-Schlitten-Platte
\end{itemize}

\newpage
\subsubsection{Schritt 1: Aufbau der Y-Achse}
Der obig verlinkten Anleitung folgend wurde beim Aufbau des Druckers zuerst die Y-Achse zusammen gebaut.
Hierfür wurden anfangs die nötigen Schrauben und Verbinder-Teile auf die zwei M8-Gewindestangen geschraubt:
\includegraphics[clip=true,trim=0 100 0 40,width=\textwidth]{Bilder/Y_Assembly_Tutorial_1.jpg}

Als nächstes wurden die zwei Linear-Stangen zusammen mit den Y-Endplatten festgeschraubt. Wichtig ist, dass die Linearlager schon vorher auf die Stangen gesteckt werden!\\
\includegraphics[width=\textwidth]{Bilder/Y_Assembly_Tutorial_2.jpg}

Anschließend wurde der Y-Schrittmotor, sowie die Riemenführung, an das nun fertige Y-Frame angebracht:\\
\includegraphics[clip=true, trim=30 0 100 0, angle=90, width=\textwidth]{Bilder/Y_Assembly_1.jpg}

Zum Abschluss wurde nun die Heizbett-Halterung mithilfe von Kabelbindern auf die Linearlager der Y-Achse angebracht, und der GT2-Treibriemen mit Bett-Halterung und Motor verbunden.\\
\includegraphics[clip=true, trim=0 40 0 40, angle = 0, width=\textwidth]{Bilder/Y_Assembly_Bed.jpg}

\subsubsection{Schritt 2: Aufbau der Y-Z-Einheit}
Nachdem die Basis des Gestells, die Y-Achse, fertig gebaut war, konnte angefangen werden die Z-Achse auf zu bauen. Hierzu wurde die Hauptplatte an die vorher auf die Y-Achse geschraubten Verbinder-Teile angeschraubt. Wichtig war hierbei ein korrekter Abstand zu der hinteren Y-Achsen-Platte, um einen geraden Aufbau des Druckers zu gewährleisten. Dies konnte mithilfe einiger weiterer Acryl-Teile erreicht werden, welche danach angebracht wurden.\\
\includegraphics[clip=true, trim= 0 0 0 0, angle=0, width=\textwidth]{Bilder/Z_Assembly_1.jpg}

Nun wurden die vorher erwähnten Teile, zwei verbindende Seiten-Platten, zwischen die Hauptplatte und der hinteren Y-Achsen-Platte geschraubt. Wichtig war, dass auf die Ausrichtung geachtet wurde, um die Platte für die Stromversorgung auf die rechte Seite des Druckers zu bringen:\\
\includegraphics[width=\textwidth]{Bilder/Z_Assembly_2.jpg}
Die Verbindungen zwischen beiden Platten wurde mithilfe von M3-Schrauben und Muttern erledigt.

\subsubsection{Schritt 3: Aufbau der Z-Motoren}
Mit der Hauptplatte fest und gerade angebracht konnte nun die Z-Achse befestigt werden. Dies beinhaltet den Aufbau der Motor-Halterungen, sowie die zwei Linearachsen und die Gewindestangen, mit welchen der Druckkopf angehoben oder gesenkt werden kann.

Zuerst wurden hierfür mithilfe einiger Acryl-Teile die Motorhalterungen an die Hauptplatte angebracht.
Wichtig war hierbei zu beachten, dass eine Seite eine Halterung für einen Z-Endstop hatte, die andere jedoch nicht!\\
\includegraphics[clip=true, trim=30 100 0 0, width=\textwidth]{Bilder/Z_Assembly_3.jpg}

In diese Halterungen wurden nun die Motoren eingebaut. Wichtig war, dass die Motor-Kabel durch die dafür vorgesehenen Führungslöcher geleitet wurden!\\
\begin{center}
\includegraphics[clip=true, trim=30 60 30 130, width=0.7\textwidth]{Bilder/Z_Assembly_4.jpg}
\end{center}

\subsubsection{Schritt 4: Aufbau der X-Achse}
Bevor die Z-Achse fertig gestellt werden kann, muss zuerst die X-Achse aufgebaut werden, da diese später auf die Z-Achsen geschoben werden muss, bevor diese zusammen geschraubt wird!! 
Wichtig ist auch zu bemerken: Im Online-Tutorial wurden für die X-Achsen-Teile 3D-Gedruckte Teile verwendet. Der Schul-Drucker hat jene Teile nicht, stattdessen wurde er komplett aus Aluminium-Teilen gefertigt!

Die X-Achse wurde dazu wie folgt aufgebaut:
Die zwei Linearlager wurden auf die Linearstangen geschoben, zusammen mit zwei Befestigungsmuttern. Hiernach wurden die zwei X-Achsen-Befestigungen auf die Seiten der Stangen aufgeschoben. An ihnen beiden wurden jeweils ein Linearlager und eine spezielle Gewindemutter mit Befestigungsflansch angebracht, zusätzlich wurde an dem linken Endstück die Riemenführrolle befestigt.
Das Endergebnis sollte dementsprechend so aussehen:\\
\includegraphics[clip=true, trim=0 0 0 100, width=\textwidth]{Bilder/X_Assembly_1.jpg}

%\paragraph{Anmerkung:} Im Folgenden unterscheiden sich Online-Anleitung und der schulische Drucker stark. Z.B. erfordert die Online-Anleitung ein Zusammenbau des Extruders. Dies war für den Schul-Drucker nicht nötig. Dennoch sind viele der Schritte sehr ähnlich.\\
Zuerst wurde der X-Schlitten mithilfe von Kabelbindern auf die X-Achse angebracht:\\
\includegraphics[clip=true, width=\textwidth]{Bilder/X_Assembly_2.jpg}