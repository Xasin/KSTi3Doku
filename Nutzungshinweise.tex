Die Verwendung eines 3D-Druckers erfordert gewisse Grundkenntnisse des Systems und dessen Möglichkeiten, und ist nicht immer einfach. Aus diesem Grund ist es gebeten, bei eventuellen Fragen oder Unklarheiten etwas erfahrenere Benutzer an zu sprechen. Es ist zudem nicht gestattet den Drucker ohne die nötigen Kenntnisse (d.h. Erwerb einer Lizenz zum Drucken), oder während sich dieser in einem fehlerhaften Zustand (bei Fehlermeldungen, ungewöhnlichem Verhalten etc.) befindet.

Zudem sollte folgendes beachtet werden: 
\begin{itemize}
\item Die Verwendung des Druckers geht auch mit der Verantwortung einher, den Druck in regelmäßigen Abständen zu überwachen, um eine fatale Fehlfunktion rechtzeitig zu erkennen und Schaden zu vermeiden, bzw. um sicher zu stellen dass der eigene Druck nicht fehlgeschlagen ist.

\item Bei der Verwendung des Druckers, sowie beim Design des Modells, sollte immer mit gesundem Menschenverstand gearbeitet werden. Zwar besitzt die Maschine Schutzregulierungen, diese können jedoch nicht jeden Fehler abfangen. Modelle, die in der Luft gedruckt werden würden, oder das Drucken ohne Filament im Extruder sind strengstens zu vermeiden! 

\item Das Wechseln des Filamentes ist nur Leuten erlaubt, welche sich mit dem Umgang mit der Maschine auch gut auskennen. Beim Wechseln, vor allem zu exotischen Materialien hin oder von diese Weg, können Fehler auf treten, die das Drucken erschweren bzw. fehlschlagen lassen! 
Dies gilt auch für den Fall dass das momentan eingelegte Filament kurz davor ist aus zu laufen, oder man mit einem anderen Material drucken möchte! Immer jemand dafür geeigneten ansprechen!
\end{itemize}

