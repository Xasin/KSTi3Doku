\label{ch:DesignChoices}
Beim Designen eines Modells muss auf vielerlei geachtet werden. Auch ist eine gewisse Grundkenntnis der Fähigkeiten des Druckers von großem Vorteil, da schon kleinste Änderungen zur Verbesserung, aber auch zur Verschlechterung der Druckbarkeit führen können.

\paragraph{Auf Ästhetik} sollte nur am Ende des Designprozesses Wert gelegt werden. Die Druckbarkeit eines Modells hängt stark von wichtigen Eigenschaften wie feinen Details oder Rundungen ab. Oftmals ist ein plumpes, eckiges Modell viel einfacher zu drucken als ein gut aussehendes, abgerundetes Modell.

\paragraph{Bodenhaftung} ist bei \emph{jedem} Modell, und mit jedem Material, wichtig. Es ist beinahe unmöglich ein Objekt zu drucken, welches nicht "`von früh an"' auf eine bestimmte Ausrichtung hin geplant wurde.
Die Seite, welche auf die Glasplatte gedruckt wird, sollte möglichst fest und groß sein. Es sollten auch, wenn möglich, nur vertikale Wände von dieser Seite aus ausgehen, da sich diese sonst viel leichter hoch biegen, und den Druck vereiteln.

\paragraph{Wände} dürfen gern so dick sein wie sie wollen. So dünn wie möglich geht allerdings nicht. Eine Wand die druckbar sein soll sollte mindestens 1mm dick sein, zu klein und sie wird nicht stabil genug, oder der Drucker wird sie einfach nicht ausdrucken können!

\paragraph{Feine Details} erliegen dem gleichen Problem. 3D-Druck ist keine zu präzise Angelegenheit, und Details wie Schriftzüge, die kleiner als 1.5mm sind können eventuell nicht so wie erwartet raus kommen.
Grundsätzlich ist es deshalb besser, \emph{ohne} Feinheiten auskommen zu können. Die Prämisse: Lieber zu dick als zu dünn.

\paragraph{Brücken und Winkel} sind eine schwierige Angelegenheit für jeden FDM Drucker, also auch unseren.
Als Brücke wird ein Stück Modell bezeichnet, welches horizontal liegt, und frei in der Luft hängt, wobei es von beiden Seiten mit einem Stück "`gestütztem"' Modell verbunden ist. Vermeiden muss man diese nicht, allerdings sollte man sich daran erinnern dass diese Stücke meist ein wenig "`herunter sacken"', da sollte man also nicht auf zu hohe Präzision hoffen.
Winkel sind alle Wände, die mehr als 45° von der Horizontalen weg gedreht sind. Je kleiner der Winkel desto schwieriger werden sie zu drucken, empfehlenswert ist es, nicht tiefer als 70° zu gehen. Generell darf man auch hier nicht auf extrem gute Präzision hoffen, jedoch ist alles bis etwa 60° bedenkenlos druckbar - je nach Druckprofil.

\paragraph{Overhangs} sind ähnlich wie Brücken, jedoch nur an \emph{einer} von beiden Seiten befestigt. Das sorgt dafür dass sie ab einer Länge von 3mm quasi unmöglich sind, sauber zu drucken.
Man sollte sie \emph{wenn möglich immer vermeiden!}