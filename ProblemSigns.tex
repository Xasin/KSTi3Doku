\subsubsection[Kein Filamentaustritt]{Kein Filament tritt aus - Extruder macht klickende Geräusche}

Dieses Problem kann auftreten, wenn die Düse des Extruders verstopft ist, oder das Filament anderweitig nicht ordnungsgemäß extrudiert werden kann. Der Extrudermotor kann sich in diesem Fall nicht mehr drehen und fängt an zu "`springen"'.

Sollte dieser Fehler auf der ersten Schicht einmalig auftreten ist dies kein Problem für den weiteren Druck. Die Düse ist in diesem Fall zu nah an der Glasplatte, dies ist jedoch normal. 
Ein Personal sollte dennoch darauf benachrichtigt werden, um das Problem zu beheben. \\
\emph{Vorsicht! Es könnte auch ein Fehler sein!} Es ist auf jeden Fall ratsam den Druck ein wenig weiter zu beobachten, um zu gewährleisten dass der Druck ohne zu große Probleme fortlaufen kann!

Sollte dieses Problem auf einer anderen Schicht auftreten, ist der Druck \emph{sofort} zu unterbrechen und der Drucker ab zu stellen. Ein Personal sollte umgehend benachrichtigt werden, um den Fehler schnellstmöglich zu beheben. Ein fortfahren des Druckers oder erneuter Versuch ist \emph{nicht möglich.}

\subsubsection{Druck "`verrutscht"' seitlich}

Im Laufe längerer oder komplexerer Drucke kann es vorkommen dass der Druck ab einer gewissen Schicht plötzlich seitlich "`verrutscht"' ist. Dies kann dadurch passieren, dass entweder einer der Motor-Treiber überhitzt ist, oder der Druckkopf an einem Detail des Druckes hängen geblieben ist und somit aus seiner bekannten Position gebracht wurde.

Normalerweise sollte dieser Fehler kein zweites mal auftreten, der Druck kann dementsprechend neu gestartet werden um es erneut zu versuchen.

Sollte jedoch dieses Problem mehrfach auftreten sollte der Drucker abgeschaltet und Personal benachrichtigt werden, um ein eventuelles mechanisches oder technisches Problem beheben zu können.

\subsubsection{Motoren drehen nicht, machen laute Geräusche oder gar nichts}

Dies ist ein mögliches Anzeichen für defekte Schrittmotortreiber. Diese können durch Kurzschlüsse, statische Aufladung oder Spannungsspitzen beschädigt werden, und können dementsprechend den Motor nur noch teilweise oder gar nicht mehr steuern.

Es ist empfehlenswert die Maschine \emph{sofort} aus zu schalten, da ein defekter Schrittmotortreiber teilweise intern kurzgeschlossen sein könnte, und dies andere Elemente gefährdet! Es sollte der defekte Treiber umgehend durch ein ähnliches, pin-kompatibles Modell ersetzt werden, bevor der Drucker erneut in Betrieb genommen werden kann!
Geeignet sind hierfür: Pololu/Allegro A4988, Trinamic TMC2100 (SilentStepStick), Texas Instruments DRV8825