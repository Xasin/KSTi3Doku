Der Drucker besteht aus folgenden wichtigen Bausteinen: 

\subsubsection{Hardware}

\paragraph{Typ:}
Der Schuldrucker ist ein GeeeTech Prusa i3 mit 8mm Acryl-Frame, gekauft von der GeeeTech e-Bay Website.

\paragraph{Motoren:} 
Die verbauten Schrittmotoren sind generische NEMA-17 Bipolar-Schrittmotoren. Getrieben werden sie von einem Pololu A4988 Schrittmotortreiber mit 1/16 Microstepping.

Folgende Steps/MM-Werte liegen vor:
\begin{itemize}[noitemsep]
\item X/Y: 80
\item Z:   2560
\item E:   93
\end{itemize}

\paragraph{Steuerelektronik:}
Die momentane Elektronik ist das GeeeTech GT2560. Die Schaltkreise basieren auf den Standard Ultimaker-Controllern, und sind Firmware-Compatibel mit diesen.

Zusätzlich wurde für eine problemlose Ansteuerung des Heizbettes ein selbst-gelöteter Schaltkreis eingebaut.
Dieser besteht aus einem Hoch-Strom MOSFET für Ströme bis zu 20A, mit geeigneter Kühlung und Schaltelektronik.

\paragraph{Firmware:}
Die Firmware der Elektronik ist (momentan) Marlin 1.0.2, jedoch kann diese theoretisch jederzeit auf eine andere RAMPS-Kompatible Version geändert werden! (So z.B. Repetier)

\paragraph{Stromversorgung:}
Die Stromversorgung des Druckers besteht aus einem generischen 12V 20A Netzteil.

\paragraph{Extruder:}
Der Extruder ist momentan ein Direct-Drive Extruder (d.h. ohne Zahnradsystem), PTFE-Lined, somit max. Temperatur 270°C (empfohlen jedoch <240°C!!!)
\begin{labeling}{Heizelement:}
\item[Thermistor:] Der Thermistor (Temperatursensor) ist ein EPCOS B57560G1104F, ausgelegt für Temperaturen bis 300°C
\item[Heizelement:] Das Heizelement ist eine 30W 12V Standard-Heizpatrone (6x20mm Größe)
\item[Düse:] Die Düse ist eine Stainless-Steel (kratzfeste) 0.4mm Düse (austauschbar), geeignet zum Druck von exotischen und abrasiven Materialien
\end{labeling}

\subsubsection{Maximalwerte}
Der Drucker unterliegt beim Drucken folgenden Limitationen:

\begin{labeling}{Geschwindigkeit:}
\item[Temperatur:] Die Temperatur darf 270°C nicht überschreiten. Das Heizbett kann max. circa 110°C erreichen.
\item[Geschwindigkeit:] Die Z-Achse kann max. 4mm/s schnell sein. 
X und Y-Achsen können sich mit max. 180mm/s bewegen.
\item[Extrusion:] Der Extruder sollte nicht mehr als 4mm$^3$/s Extrudieren müssen.
\item[Druckfläche:] Die Druckfläche des Druckers liegt bei ca. 18x18x18cm
\end{labeling}